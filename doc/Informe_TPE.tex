\documentclass[a4paper,14.5pt]{article}
\usepackage[latin1]{inputenc}
\usepackage{t1enc}
\usepackage[pdftex]{graphicx}
\usepackage{multicol}
\usepackage{fancybox}
\usepackage{fancyhdr}
\usepackage[pdftex,usenames,dvipsnames]{color}
\usepackage{listings}
\pagenumbering{arabic}
\pagestyle{fancy}
\fancyhead[L]{POO}
\fancyhead[C]{curso 2011}
\fancyhead[R]{TPE cuat 1}
\title{\Huge{Programaci�n orientada a objetos \\TPE \\Dungeon Game}}
\begin{document}

\maketitle
\vspace{70mm}
\large{\underline{Autores:}} 
\begin{center}
  \begin{tabular}{l r}
    \emph{\author{Tom�s Mehdi}} & Legajo: \emph{51014}\\
    \emph{\author{Alan Pomerantz}} & Legajo: \emph{51233}\\
  \end{tabular}
\end{center}
\pagebreak

\section{Desarrollo y justificaciones del dise�o e implementaci�n del juego}
Cuando comenzamos a hablar sobre el dise�o surgieron varias ideas, a pesar de las variadas ideas, 
las buenas opciones para un buen dise�o orientado a objetos eran pocas. 
Para comenzar se decidio hacer una jerarquia para los monstruos y los 
jugadores, estas dos clases heredan de "Character"; la cual es una clase 
abstracta. La justificacion a esta jerarquia es que monstruo y player tienen 
comportamientos parecidos y muchas variables en comun, pero no son lo mismo ya 
que el player puede realizar algunas cosas mas que un monstruo.
Se creo la interfase "Putable" lo que nos permite tratar a todas 
las cosas que pueden ponerse en un tablero de la misma manera. Por 
cuestiones de implementacion se creo tambi�n la clase abstracta Cell
 la cual tiene una variable de instancia "isVisible" y los getters y 
setters de la misma. Esto facilito la opcion del juego de estar todo 
escondido hasta que el jugador lo descubre. A esta clase la heredan 
todos las clases que pueden ser insertadas. Aclaro que hice extender 
a Character de Cell y no al mismo Monster, porque existe la posibilidad
 de en un futuro en ves de solo poner monstruos escondidos tambien poner
 jugadores escondidos. La otra opcion era implementar directamente los 
metodos en la clase Monster, pero parecio mejor la idea de que un Character 
pueda ser visible o no.
Se realizo un parseo con 
una gran orientacion a objetos, lo cual permitio utilizar esa clase para 
realizar la carga de un juego luego de ser guardado. No hay mucho que acotar 
a esta implementacion, hay partes imperativas, no hay forma de evitarlo, pero 
se realizo de una forma ordenada.
El front se basa en los metodos del "DungeonGameListener", lo cuales son implementados 
en "DungeonGameFrame" en una clase inner privada "DungeonGameListenerImp". "DungeonGameFrame" 
hereda de "GrameFrame", una clase abstracta que podria utilizarse para crear el frame de distintos 
juegos. "DungeonGameFrame" contiene una instancia de un "DataPanel" y una de "DungeonPanel". Solo el 
Dungeon Panel hereda de la clase "GamePanel" otorgada por la catedra. La opcion de pasar el mouse por 
encima de un character en estado visible y que aparesca en el DataPanel complico un poco las cosas. Se 
creo una clase inner a "DungeonGameFrame" llamada "DungeonPanelListener", en la cual se aplican los eventos 
a las diferentes posisciones del mouse. Esta jerarquia ademas de ser muy logica favorece mucho la implementacion, 
ya que se reutiliza mucho codigo y simplifica la creacion de la parte grafica con JavaSwing.
El topico de guardado y cargue del un juego al principio parecia que debia ser una cualidad del juego, "el juego 
sabe guardarse y cargarse", pero luego descubrimos que si no era una "cualidad" del juego podria modularizarse 
para guardarse como cada uno quisiera sin tener que modificar la clase juego sino creando una propia clase para
 el guardado del juego. Lo cual nos guio a crear una interfase e implementar nuestra propia clase que la implementara.

\end{document}